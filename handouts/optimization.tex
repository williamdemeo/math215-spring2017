%%% Uncomment only one of the next two lines (depending on whether you want the answers printed)
\documentclass[fleqn,addpoints,12pt]{exam}
%%\documentclass[answers,addpoints,12pt]{exam}

\usepackage[usenames, dvipsnames]{color} % defines a new color
%\definecolor{SolutionColor}{rgb}{0.8,0.9,1} % light blue

\renewcommand{\solutiontitle}{\noindent\textbf{Answer: }}

%% How to print correct answer choices:
   \CorrectChoiceEmphasis{\itshape\bfseries} %% <-- bold italics
%%    
%% \CorrectChoiceEmphasis{\color{ForestGreen}\bfseries} %% <-- green bold

\pointsinmargin
\pointpoints{pt}{pts}
\marginpointname{pts}

\makeatletter
\newif\ifanswers
\@ifclasswith{exam}{answers}{\answerstrue}{\answersfalse}
\makeatother
\newcommand{\scratchpage}{%
  \ifanswers % do nothing
  \else \newpage \thispagestyle{empty} \begin{center} -- scratch -- \end{center} \fi}

\newcommand{\foo}{\ifanswers fooone\else footwo\fi}

%% Change geometry if you want:
%% \usepackage[top=2cm, left=2cm,right=2cm,bottom=1cm]{geometry}%

\usepackage{amsmath}
\usepackage{amsthm,amssymb}
\usepackage{mathtools}
\usepackage{url,multicol,enumerate}
\usepackage{tikz}

\theoremstyle{remark}
\newtheorem{theorem}{Theorem}
\newtheorem*{prop}{Proposition}
\newtheorem{problem}{Problem}
\newtheorem*{prob}{Problem}
\newtheorem*{answer}{{\bf Answer}}
\newtheorem*{answers}{{\bf Answers}}
\newtheorem*{explanation}{{\bf Explanation}}
\newtheorem*{hint}{{\it Hint}}
\newtheorem*{ex}{Exercise}


%% Some of my own personal favoriate macros... (remove these if you want)
\renewcommand{\vec}[1]{\mathbf{#1}}
%%       To make a boldface vector, use backslash v in front of the 
%%       letter and add a new command for that letter here or in 
%%       the macros.tex file:
\newcommand\rank{\ensuremath{\operatorname{rank}}}
\newcommand\nullity{\ensuremath{\operatorname{nullity}}}
\newcommand{\<}{\ensuremath{\langle}}
\renewcommand{\>}{\ensuremath{\rangle}}
\newcommand{\ur}{\ensuremath{\underline{\mathrm{r}}}}
\newcommand{\uT}{\ensuremath{\underline{\mathrm{T}}}}
\newcommand{\uF}{\ensuremath{\underline{\mathrm{F}}}}
\newcommand{\uN}{\ensuremath{\underline{\mathrm{N}}}}
\newcommand{\ui}{\ensuremath{\underline{\mathrm{i}}}}
\newcommand{\uj}{\ensuremath{\underline{\mathrm{j}}}}
\newcommand{\ua}{\ensuremath{\underline{\mathrm{a}}}}
\newcommand{\ub}{\ensuremath{\underline{\mathrm{b}}}}
\newcommand{\un}{\ensuremath{\underline{\mathrm{n}}}}
\newcommand{\uv}{\ensuremath{\underline{\mathrm{v}}}}
\newcommand{\R}{\ensuremath{\mathbb{R}}}
\newcommand\va{\vec{a}}
\newcommand\vb{\vec{b}}
\newcommand\vu{\vec{u}}
\newcommand\vv{\vec{v}}
\newcommand\vw{\vec{w}}
\newcommand\vs{\vec{s}}
\newcommand\vx{\vec{x}}
\newcommand\vy{\vec{y}}
\newcommand\vz{\vec{z}}
\newcommand\vzero{\vec{0}}


\newcommand{\dotsize}{1pt}
\newcommand{\Heq}{\ensuremath{ \; \stackrel{\mathrm{H}}{=}} \; }

\pagestyle{foot}
%%% Running footer will have a space for page score (if this is not the solution key)
% \ifanswers  %% do nothing
% \else
% \runningfooter{}{}{Score for this page: \makebox[1in]{\hrulefill} out of \pointsonpage{\thepage}}
% \fi


\begin{document}

\noindent {\bf Optimization and Related Rates Examples}
\hfill \ifanswers {\bf ANSWERS}\fi

\newcommand\probskip{\vskip2cm}

\bigskip

\newcommand\mcval{3}

\begin{questions} % Begins the questions environment

  %%%% Problem 1.
  %%%%%%%%%%%%%%%%%%%%%%%%%%%%%%%%%%%%%%%%%%%%%%%%%%%%%%%%%%%%%% 
  \question After acing your calculus exam, you decide to apply what you have learned.
  You move to Colorado and start a marijuana farm.  You have \$8000 to spend on 
  an enclosure for a rectangular garden. 
  Along one side of the garden is a brick wall (which you don't have to pay
  for).  Two sides of the garden will be perpendicular to the wall and 
  made of wood fencing which costs \$20 per foot. The side parallel to the wall
  will be made of chain link fence which costs \$10 per foot.  If you have
  \$8000 to spend on the enclosure, what dimensions will maximize the area of the garden?

  (Hint: draw a picture; write down an area function, which you want to 
  maximize; write down a cost function, which will equal 8000.) 


  \vfill

  \hfill length of perpendicular side: \phantom{XXXXXXXXXXXX} ft\\
  \phantom{XX} ~ \hfill  \underline{\phantom{XXXXXXXXXXXXX}}
  \vskip1cm
  \hfill length of parallel side: \phantom{XXXXXXXXXXXX}ft\\
  \phantom{XX} ~ \hfill  \underline{\phantom{XXXXXXXXXXXXX}}
  \vskip1cm
  \hfill total area: \phantom{XXXXXXXXXXXX} ft$^2$\\
  \phantom{XX} ~ \hfill  \underline{\phantom{XXXXXXXXXXXXX}}
\\
\hrule
\hskip-.7cm  Author: William DeMeo \url{williamdemeo@gmail.com}

\hskip-.7cm Updated: March 15, 2017
  \newpage
  %%%% Problem 2.
  %%%%%%%%%%%%%%%%%%%%%%%%%%%%%%%%%%%%%%%%%%%%%%%%%%%%%%%%%%%%%% 
  \question
  \begin{minipage}[t]{0.35\textwidth}
    According to a recent study, Def Jam Records spent \$1,078,000
    producing Rihanna's single ``Man Down.''
    % (\$78,000 for musicians and
    % songwriters, and \$1,000,000 for marketing and promotion of the single).

    Generally speaking, the demand for song downloads goes down as 
    the price per download goes up. 
    Suppose the quantity $q$ of downloads demanded is given by the
    following function of price:
    \begin{equation}
      \label{eq:1}
      q = \frac{12000000}{p^2+9}.
    \end{equation}
    The revenue generated from selling $q$ downloads at price $p$
    (dollars) is price times quantity: $R(p) = qp$.
  \end{minipage}
  % second column
  \begin{minipage}[t]{0.65\textwidth}
    \vskip-4mm
    \begin{center}
      \includegraphics[height=3.5in]{RihannaCost}
    \end{center}
  \end{minipage}

  \bigskip

  What price should Def Jam charge for each download in order to
  maximize revenue? ({\it Hint:} Use Eq.~(\ref{eq:1}) to express 
  revenue as a function of price only, then differentiate.)

  \ifthenelse{\boolean{answers}}{\\[4pt]Answer: (b) \$1.}{}
  %% \item Find the absolute maximum and absolute minimum values of
  %%   $f(x) = x^3 - 3x + 1$ on the interval $[0,3]$.

  \newpage
  %%%% Problem 3.
  %%%%%%%%%%%%%%%%%%%%%%%%%%%%%%%%%%%%%%%%%%%%%%%%%%%%%%%%%%%%%% 

  \question
  Each side of a square is increasing at a rate of 2 cm/s. At what rate is
  the area of the square increasing when the area of the square is 49
  $\mathrm{cm}^2$?  (Include appropriate units.)

  \vfill

  \hfill {\bf Answer:} $A'= $ \phantom{XXXXXXXXXXXXXX}\\%$\mathrm{cm}^2/\mathrm{s}$ \\
  \phantom{XX} ~ \hfill  \underline{\phantom{XXXXXXXXXXXXXXX}}

  \bigskip

  \newpage
  %%%% Problem 4.
  %%%%%%%%%%%%%%%%%%%%%%%%%%%%%%%%%%%%%%%%%%%%%%%%%%%%%%%%%%%%%% 
  \question After acing your calculus exam, you are hired at the Near Earth
  Object Observatory atop a volcano on the island of Maui. Your job is to
  support the early detection system for meteors that might impact
  earth.  On your first day you discover a meteor, in the shape
  of a perfect sphere, fast approaching earth.  As it
  travels through the earth's atmosphere and burns up, its surface area
  decreases at a rate of $24 \pi$ $\mbox{m}^2$/second.
  \begin{parts}
  \part At what rate is the radius decreasing when the radius is 3 meters?\\
    (Hint: Surface area of a sphere is $S = 4 \pi r^2$; apply the chain rule.)
    \vskip5cm

  \part
    In solving Part i, you hopefully found $r'(t)$, the rate of change of the radius.
    Now write down an integral expression that gives the total change in the radius as time
    goes from $a$ seconds to $b$ seconds.

    \vskip5cm
    \part Suppose at time $t=1$ the meteor is observed to have a
    radius of 32 meters, and suppose it will reach the earth 
    after $e^{10}\approx 22026$ seconds, what will be its radius upon impact?

  \end{parts}


\end{questions}

\end{document}


























